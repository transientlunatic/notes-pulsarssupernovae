Pulsar surveys, where special attention is paid to sources which The
majority of pulsars are now discovered as part of systematic have
rapid and periodic variations in brightness temperature. These surveys
require high temporal resolution, and large collectors, so the
telescopes best suited are instruments like the Parkes telescope,
Arecibo, or Greenbank. As a result of the high resolution most surveys
are limited by the computational power available to the endeavour.

In general any pulsar, when found, will have unknown dispersion, sky
location, period, and pulse shape. Moreover, if the pulsar is in a
binary system there may be additional delays and shifts originating at
the pulsar's locality. If each patch of sky is observed only for a
short time then the dispersion, period, and pulse shape are the
significant contributors to the detection probability.

\section{Period search}
\label{sec:period-search}

The simples means of identifying the period of a pulsar signal is to
perform a Fast Fourier Transform (FFT) on the data, which concentrates
the power from the pulses into a small number of frequency bins. This
method is effective at determining the rough period of the pulsar. 

Once the period of the pulsar is identified the signal is \emph{phase
  folded}, whereby the signal from each point in the pulse phase are
added together over the entire observation period. If the period is
known precisely then the appropriate bin can be immediately identified
for each timestamp, and the integrated pulse profile will be
returned. If the period is poorly known then the pulse profile
returned by this process will be smeared out. By using variational
principles it is possible to run an algorithm to determine the optimal
fit to the period using this method.

The correct de-disperion must also be found by trial and error, so
each timeseries is corrected for a trial DM, which is searched for
harmonics before a new trial DM is tested.

\section{Timing}
\label{sec:timing}

Once a pulsar is observed it can be timed; the times-of-arrival of
each pulse can be compared to the pulses of an atomic clock. We
measure the raw TOAs topocentrically, but it is conventional to
convert these to the times as they would be measured at the solar
system barycentre, so that they are measured in the nearest thing we
have to an inertial reference frame. General relativity also affects
the timings from the atomic clocks, so best accuracies in the timings
are around $100\,\nano\second$.

\subsection{The Earth's orbit}
\label{sec:earths-orbit}

The distance between the Earth and the pulsar, and the Earth and the
solar system barycentre (SSB) varies over the period of a year, with
the Earth's orbital motion. Thanks to the non-planar nature of this
delay it affects sources differently depending on their ecliptic
coordinates in the sky. Assuming a circular orbit this delay, $t~c$,
takes the form
\begin{equation}
  \label{eq:timing-3}
  t~c = A \cos(\omega t - \lambda) \cos(\beta)
\end{equation}
for $\omega$ the angular velocity of the Earth, $\lambda$ the ecliptic
longitude, and $\beta$ the ecliptic latitude of the pulsar.

\section{Timing delays}
\label{sec:timing-delays}

The relation between the timings received at the telescope, $t$,
compared to those at the SSB, $t~b$ is
\begin{equation}
  \label{eq:4}
  t~b = t - \qty( {\cal D} \frac{\DM}{ \nu^2} + \Delta~{R \oplus} + \Delta~{E \oplus} + \Delta~{S \oplus} )
\end{equation}
with ${\cal D} \frac{\DM}{\nu^2}$ the dispersive propagation delay at
a frequency $\nu$, $\Delta~{R \oplus}$ the Roemer delay---the time
taken for the signal to travel across the solar system, $\Delta~{E
  \oplus}$ is the Einstein delay, which is the dilation effect
introduced into Earth's clocks by the mass of the solar system,
$\Delta~{S \oplus}$ is the Shapiro delay, caused by the added
spacetime geometry produced by the mass of the solar system.

\section{Spin-down model}
\label{sec:spin-down-model}

By making individual timings of pulsars it is possible to measure
their period's variation over time,
\begin{equation}
  \label{eq:114}
  P(t) = P(t_0) + \eval{\dv{P}{t}}_{t_0} (t-t_0) + \eval{ \dv[2]{P}{t}}_{t_0} \half (t-t_0)^2 + \cdots
\end{equation}
The first two coefficients, P and Pdot, are found by fitting a second
order polynomial to the pulsar timings; the higher coefficients are
small and dominated by timing noise. Figure \ref{fig:ppdot} shows the
P-Pdot diagram; the pulsar world's equivalent of the Herzprung-Russell
diagram. One feature visible on this diagram is the pulsar \emph{death
  line}, a lower limit to the spin-down luminosity, which have
insufficient energy to fuel the acceleration process which generate
the radio beams.

\section{Parallax}
\label{sec:parallax}

While the dispersion measure can be used to estimate the distance to a
pulsar, provided a good model of the electron density is available,
accurate pulsar timing allows precise parallactic measurements, by
measuring the curvature of the signal wavefront over the course of the
Earth's orbit, which allows the radius of curvature to be
calculated. This works best for nearby pulsars (within
1\,\kilo\text{pc}, and for pulsars near the ecliptic.

%%% Local Variables: 
%%% mode: pdflatex
%%% TeX-master: "../project"
%%% End: 
