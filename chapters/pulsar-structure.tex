Neutron stars have a structure which is heavily dependent upon their
equation of state, something which is poorly understood at
present. All observed neutron stars seem to have a mass close to the
Chandrasekhar mass limit, and they have an upper limit set by the
Oppenheimer-Volkov limit, which is probably around $2\,M_{\odot}$.

\section{Formation of neutron stars}
\label{sec:form-neutr-stars}

Neutron stars are almost always left over from core-collapse
supernovae, although it is possible some may be produced from white
dwarf collapse in type Ia supernovae. Supernovae are believed to occur
every 65 years in the Galaxy, and the oldest observed remnant has
lasted around $10^5$ years, whereas the majority of pulsars are older
than $10^6$ years. Neutron stars do not remain within the remnant, but
tend to travel at high speeds due to asymmetrical kicks produced in
the supernova explosion.

\section{Internal Structure}
\label{sec:internal-structure}

%\begin{figure*}[b!]
%  \centering

%  \caption{A schematic view of the interior of a neutron star.}
%  \label{fig:neutron-interior}
%\end{figure*}

The average mass density in a neutron star is around $\rho =
6.7\e{17}\,\kilogram/\meter^3$, which is greater than twice the
nuclear density, and the actual density increases towards the
centre. The observation of glitches implies than neutron stars have
superfluid interiors, and spectra imply they have an iron crust, in a
close-packed lattice, and with degenerate electrons. Here $\rho\approx
10^9\,\kilogram/m^3$. This solid crust is only around
$500\,\kilo\meter$ thick, and contains only $1$ to $2\%$ of the total
moment of intertia. There is also a dense atmosphere around
$0.5\,\centi\meter$ thick. Observations of gravitational waves imply
that the surface must be highly homogeneous, with a maximum potential
deviation smaller than $0.1\,milli\meter$. The neutron star is,
however, oblate due to its extreme rotational energy, with a
flattening of between $10^{-2}$ and $10^{-8}$.

   \begin{tikzpicture}[node distance=1.5cm]

	\foreach \x in {0, 1, ..., 16}
		\draw (\x,1.2) node {\x};

	\node at (15.5, -0.3) {$4.3\,\times 10^{11}$};
	\node at (12, -0.3) {$\sim 2\,\times 10^{14}$};
	\node at (1, -0.3) {$4.4\,\times 10^{14}$};
	\node at (7, -0.3) {Density [g/cm$^3$]};
	\node [fill=white] at (7, 1.2) {Radius  [km]};

	\filldraw [fill=accent-red!10, draw=black, thick] (16,0) rectangle (15,1)
	node [right=1.8cm, below=1.8cm, text width=3.5cm] {Outer crust.\\ Nuclei \& electrons.}
	;

	\filldraw [fill=accent-red!30, draw=black, thick] (15,0) rectangle (12,1)
	node [right=3.8cm, text width=7.5cm, above=.9cm] {Inner crust. \\ Superfluid neutrons, nuclei, \& electrons.}
	;

	\filldraw [fill=accent-red!50, draw=black, thick] (12,0) rectangle (11,1)
	node [right=1.8cm, below=1.8cm, text width=3.5cm] {Outer core. \\Normal neutrons.};	
	;

	\filldraw [fill=accent-red, draw=black, thick] (11,0) rectangle (0,1)
	node [right=3.8cm, text width=7.5cm, above=.9cm] {Inner core. \\ Superfluid neutrons, superconducting protons, and electrons.}
	;

%\draw (0, 4) -- (16,4);

\end{tikzpicture}

%%% Local Variables: 
%%% mode: latex
%%% TeX-master: "../project"
%%% End: 


Beneath the surface a neutron star's pressure increases, and inverse
beta decay and neutron drip dissolves the atomic structure into a
superfluid of neutrons, with around 5\% of the total structure
electrons and protons. The core is believed to consist of superfluid
neutrons and superconducting protons and electrons.

\section{Neutron drip}
\label{sec:neutron-drip}

Neutron drip is a phenomenon occurring in heavy nuclei where the
number of neutrons exceeds the number of protons. Nuclei where $Z<N$
tend to be unstable, and so neutrons ``drip'' out of them. In the high
pressure and density environment of the neutron star interior iron is
compressed so strongly that atoms become neutron rich due to inverse
beta decay, and this drip occurs.

\section{Vortex pinning}
\label{sec:vortex-pinning}

Vortex pinning is a phenomenon associated with the neutron superfluid
of the neutron star's interior.  There is no viscous damping in a
superfluid, so vorticity is perfectly conserved, and so vortex tubes
form, which each carry one $\hbar$ of angular momentum, and have a
cross section of $10^{-14}\,\meter$. Thus the area density of these
vortices defines the local rotation rate of the material. The magnetic
field is rotationally coupled to the core and the crust.

As the neutron star spins down the vortices move outwards and diffuse
into the non-superfluidic surface, allowing angular momentum to be
lost. The centre is however pinned, so can never reach the surface,
and so does not spin down with the rest of the star. Pinned vortices
will release their energy suddenly, in the form of pulsar glitches,
where the spin-rate of the pulsar increases.

\section{Radio glitches}
\label{sec:radio-glitches}

A radio glitch is an observed step increase in the rotation rate of a
pulsar, and these are normally associated with vortex unpinning from
the core, increasing the effective moment of inertia of the star. In
the Crab pulsar however, these appear to be caused by
starquakes---tectonic reshaping of the crust. In magnetars these
produce bursts of x-rays and gamma rays.



%%% Local Variables: 
%%% mode: latex
%%% TeX-master: "../project"
%%% End: 
 