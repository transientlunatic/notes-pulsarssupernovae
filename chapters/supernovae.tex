A supernova is the explosion which occurs at the end of the lifespan
of a star, which produces a sufficient quantity of energy to outshine
its host galaxy for a brief time. There are other mechanisms for the
ignition of a supernova, however.

\begin{table*}
\centering
\begin{tabular}{l ll l}
\toprule
Main Class                & Sub-class & Features                                                           & Mechanism                      \\ 
\midrule
  \multirow{3}{*}{Type I} & Ia        & Si II line.                                                        & Thermal runaway                \\
                          & Ib        & Weak or no Si, He I line.                                          & \multirow{2}{*}{Core collapse} \\
                          & Ic        & Weak or no Si, no He.                                              &                                \\
\midrule
\multirow{4}{*}{Type 2}   & II-P      & Type II spectrum, No narrow lines, light curve plateaus.           & \multirow{4}{*}{Core collapse} \\
                          & II-L      & Type II spectrum, no narrow lines, light curve decreases linearly. &                                \\
                          & II-n      & Type II spectrum, some narrow lines.                               &                                \\
                          & II-b      & Spectrum changes to become Type Ib-like spectrum.                  &                                \\
\bottomrule
\end{tabular}
\caption{A summary of the classifications of supernova.}
\label{tab:supernova-category}
\end{table*}

There are a number of different classifications of supernovae (these
are summarised in table \ref{tab:supernova-category}), with the
subdivisions made based on spectral properties of the explosion.

\section{Type I }
\label{sec:type-i-}

If a supernova does not show Balmer lines from hydrogen it is
categorised as a type I supernova, and then sub-divided based on the
silicon and helium lines.

\subsection{Type-Ia}
\label{sec:type-ia}

Type Ia supernovae show helium and silicon ion absorption lines, and
are believed to arise as a result of mass accretion onto a white dwarf
in a binary system. This causes the white dwarf to exceed the
Chandrasekhar mass limit, and collapse to a neutron star. The
consensus view, however, is that for a period of just 1000 years, when
the white dwarf has around 99\% of the limit mass, convection occurs,
allowing deflagration through carbon and later oxygen
fusion. Degeneracy pressure is independent of temperature, and so the
fusion process is prone to runaway, and Rayleigh-Taylor instability
causes the burning surface to accelerate dramatically.

A second mechanism arises in systems with two degenerate white dwarfs,
which eventually coalesce, forming a super-Chandrasekhar star, which
then collapses to form a neutron star.

\subsection{Types-Ib and -Ic}
\label{sec:types-ib-ic}

Types-Ib and -Ic are the result of the collapse of massive stars which
have undergone a red-giant phase, and have lost most of their outer
layer of hydrogen and helium gas. They have a low occurrence rate
thanks to the scarcity of massive stars in the Galaxy.

\section{Type II}
\label{sec:type-ii}

Type II supernovae are the ``normal'' supernovae, for progenitors with
8 to 40 times the mass of the Sun, and represent the result of core
collapse, due to the inability of the progenitor to balance the
gravitational pressure of the gas by thermodynamic pressure from the
nuclear reactions in the core.

Type IIP supernovae, which show a plateau in their lightcurves, are
the result of the collapse of a star with a hydrogen envelope with
$M\ge 2M_{\odot}$, whereas IIL supernovae have less massive envelopes,
which cannot sustain the plateau.

\section{Late phase evolution}
\label{sec:late-phase-evolution}

The pressure generated in the core of a star must balance the weight
of the material composing a star to prevent collapse. In the Newtonian
regime this is described by the relation
\begin{equation}
  \label{eq:126}
  \pdv{P}{r} = - \frac{G M(r) \rho(r)}{r^2}
\end{equation}
for $P$ the pressure at a radius $r$ from the star's centre, and
$\rho(r)$ the density. In polytrophic models of stars
\[ P \propto \rho^{(n+1)/n} \]
thus
\begin{equation}
  \label{eq:127}
  \frac{P~c^3}{\rho~c^4} = 4 \pi G^3 \qty( \frac{M}{\phi} )^2
\end{equation}
for the central density and pressure, and $\phi$ the constant of
proportionality,
\begin{equation}
  \label{eq:128}
  \phi(n) =
  \begin{cases}
    4.9 & \text{ for } n=0 \\
10.7 & \text{ for } n= 1.5 \\
16.15 & \text{ for } n = 3 
  \end{cases}
\end{equation}

As a star runs out of fuel for a given fusion process it will
contract, and its temperature will increase, allowing a higher-order
fusion process to be undertaken, pausing the contraction process,
however, after helium burning the temperature in the star is
sufficient to maintain a large population of thermalised electrons and
positrons, which can annihilate to form a neutrino-antineutrino
pair. The loss of this mass accelerates the contraction process, and
so, while helium burning may last many years, silicon burning lasts
only a few weeks.

\section{Mass loss}
\label{sec:mass-loss}

The final evolution of a star is highly dependent upon the amount of
mass which is lost in the late stages of its evolution, and this is
dependent upon the star's metallicity. The mass loss is wind-driven in
large stars, and since this has a strength proportional to the square
root of the abundance of metals it becomes a dominant effect in very
metallic stars. If the star loses its entire Hydrogen envelope its
rate of mass loss can increase significantly, producing a turning
point in figure \ref{fig:remnants}, as much smaller helium cores
remain at the time of the star's collapse.


\begin{figure*}
  \centering
    \begin{tikzpicture}
    \begin{semilogxaxis}[xmin=8, xmax=300, 
                         ymin= 0, ymax=1, 
                         width=\textwidth, height=8cm,
                         xlabel=Initial mass ($M_\odot$), ylabel=log(metallicity)]
      \draw (axis cs: 5,0.5) node {1};

      \fill [muted-blue!40]
      (axis cs: 9,0) -- (axis cs: 9,1) -- (axis cs: 10, 1) -- (axis cs: 10,0) -- cycle;


      \fill [muted-blue!90]
      (axis cs: 39, 0) -- (axis cs: 39, 0.8) -- (axis cs: 300, 0.8) -- (axis cs: 300, 0) -- cycle;

      \fill [muted-blue!70]
      (axis cs: 40,0) -- (axis cs: 40, 0.6) -- (axis cs: 100,0.5) -- (axis cs: 200,0.6) -- (axis cs: 300, 0.7) -- (axis cs: 300, 1) -- (axis cs: 10, 1) -- (axis cs: 10, 0) -- cycle;


      \fill [muted-blue!20]
      (axis cs: 25,0) -- (axis cs: 25, 0.5) -- (axis cs:30,0.8) -- (axis cs: 34, 0.9)
   -- (axis cs: 80,0.8) -- (axis cs: 260,0.9) -- (axis cs: 300, 0.99) 
   -- (axis cs: 300, 1) -- (axis cs: 10, 1 ) --(axis cs:10,0);

      \fill [white] 
      (axis cs: 140,0) -- (axis cs: 260,0) -- (axis cs: 260, 0.2) -- (axis cs: 300, 0.2) -- (axis cs: 300, 0.25) -- (axis cs: 200, 0.25) -- (axis cs: 140, 0.2) -- cycle;

      \draw [ultra thick, black] (axis cs: 32, 1) -- (axis cs: 34, 0.9) -- (axis cs: 40,0.6) -- (axis cs: 120, 0.3) -- (axis cs: 100, 0.2) -- (axis cs:100, 0);
   

      \node at (axis cs: 8.5, 0.5) {\circled{1}};
      \node at (axis cs: 9.5, 0.75) {\circled{2}};
      \node at (axis cs: 15, 0.5) {\circled{3}};
      \node at (axis cs: 30, 0.5) {\circled{4}};
      \node at (axis cs: 60, 0.3) {\circled{5}};
      \node at (axis cs: 200, 0.1) {\circled{6}};
      \node at (axis cs: 280, 0.1) {\circled{7}};

      \node at (axis cs: 39, 0.8) [rotate=-65] {No H envelope.};
      \node at (axis cs: 33, 0.8) [rotate=-65] {\phantom{No }H envelope.};

    \end{semilogxaxis}
  \end{tikzpicture}
%%% Local Variables: 
%%% mode: latex
%%% TeX-master: "../project"
%%% End: 

  \caption{The remnants left by a variety of masses and metallicities.}
  \label{fig:remnants}
\end{figure*}

\section{Low-mass stars}
\label{sec:low-mass-stars}

Low mass stars, with progenitor masses lower than 9 solar masses will
shed the majority of their outer material in their red giant phase,
leaving a Carbon-Oxygen core with a mass of 0.6 to 1.1 solar masses,
which forms a white dwarf; a degenerate remnant supported by electron
degeneracy pressure. This situation is illustrated in figure
\ref{fig:remnants}: the production of a white dwarf by a star with
mass under $<9 M_\odot$ is represented by region \circled{1}. The loss
of mass in the late stages of the star's evolution are important, and
a star with a mass under 11 solar masses can still, if the element
abundances in the core are correct, form an oxygen-neon white dwarf,
thanks to the development of a convective super-wind, and this is
illustrated in region \circled{2}.

\section{High-mass stars}
\label{sec:high-mass-stars}

In high mass stars the electrons in the core do not become degenerate
until the final stages of the burn, and the luminosity of the star,
which will be close to the Eddington limit, remains almost
constant. Stars with a mass up to 100 solar masses will attempt to
fuse iron. This process is endothermic, and so fails to provide the
required pressure to counterbalance the star's gravitational collapse.
The precise remnant left behind depends on a combination of the mass
of the progenitor and its metallicity \cite{2003ApJ...591..288H}.  For
a zero-metallicity star, up to around 40 solar masses the remnant will
be a neutron star (figure \ref{fig:remnants}, section
\circled{3}). Above this, but below 100 solar masses a black hole will
form as material falls back into the core (section \circled{4}). Above
100 solar masses the collapse leads straight to a black hole, with no
supernova (section \circled{5}). Above 140 solar masses the star will
suffer pair-pair pulsational instability, and no remnant will be
produced (section \circled{6}). For extremely massive (greater than
260 solar masses, section \circled{7}) photodisintegration becomes the
dominant cause of the collapse, and the star becomes a black hole
directly, as in section \circled{5}.

\begin{figure*}[t]
  \centering
  \begin{tikzpicture}
    \begin{axis}[xlabel=$\log(T)$, ylabel=$\log(\rho)$,
                       width=0.95\textwidth, height=8cm,
                       xmin=6, xmax=10,
                       enlarge x limits=false,
                       no markers,
                       domain=5:10,
                       ymin=0, ymax=10
                       ]
      \addplot [fill, black!40, thick]{3*x-13.62-8.5} \closedcycle;
      \addplot [fill, black!30, thick]{3*x-13.62-8.5} -- (axis cs:10,10) -- (axis cs:0,10) -- (axis cs: 0,0)-- (current plot begin);
      \addplot+[fill,thick, black!10, domain=9:10] {3*x-20.5} -- (axis cs: 10,10) -- (axis cs: 0,10) -- (axis cs: 0,6.5) -- (current plot begin);
      \addplot[fill, black!20, domain=5:9]{1.5*x-7} -- (axis cs: 6,6.5) -- (current plot begin);

      \addplot [ultra thick, white, domain=0:7.2] {ln(5.7)*(exp(x)/10^6)^(-1/3) -14}
      -- (axis cs: 7.3,0) node [midway, right, white] {CNO};
      \node at (axis cs: 6.3,6) [right, white] {pp};
      \addplot [ultra thick, white] {ln(150)*(exp(x)/10^6)^(-1/3) -30};
      \node at (axis cs: 7.9,6) [right, white] {$3 \alpha$};
      \addplot [ultra thick, white] {ln(100)*(exp(x)/10^6)^(-1/2) -40};
      \node at (axis cs: 9.1, 8) [right, white]{O+O};

      \fill [black] (axis cs: 9.7,0) -- (axis cs: 9.9,0) -- (axis cs: 9.9,10) -- (axis cs:9.7,10)--cycle;
      \node at (axis cs: 9.8, 6) [rotate=-90, white] {Photodisintegration};
      \fill [black] (axis cs:8.5,0) -- (axis cs: 9.5,0) -- (axis cs: 9.5,7) -- (axis cs: 8.8, 4) -- cycle;
      \node at (axis cs: 8.8, 2.5) [white, right] {Pair instability};

      \draw [accent-red, ultra thick] (axis cs:6,0) -- (axis cs:7,3) -- (axis cs: 6,3) node [midway, above, accent-red] {$0.1\,M_\odot$};
      
      \draw [accent-red, ultra thick] (axis cs: 6.5,0) -- (axis cs:8,5) -- (axis cs:6,5) node [midway, below, accent-red] {$1\,M_\odot$};
      \draw [accent-red, ultra thick] (axis cs:7.2,0) -- (axis cs: 9.7, 7.8) node [midway, above, accent-red, rotate=22] {$10\,M_\odot$};
      \draw [accent-red, ultra thick] (axis cs: 8, 0) -- (axis cs: 8.65,2) node [midway, above, accent-red, rotate=22] {$100\,M_\odot$};

      \node at (axis cs: 6.8, 9) {UR Degenrate};
      \node at (axis cs: 6.8, 6) {NR Degenerate};
      \node at (axis cs: 6.8, 1) {Ideal gas};
      \node at (axis cs: 8.0, 1) {Radiation};

    \end{axis}
  \end{tikzpicture}
%%% Local Variables: 
%%% mode: latex
%%% TeX-master: "../project"
%%% End: 

  \caption{The $\log(T)-\log(\rho)$ diagram.}
  \label{fig:logTlogr}
\end{figure*}

\section{Collapse}
\label{sec:collapse}

If the mass of the iron core of the star exceeds the Chandrasekhar
mass limit it will no longer be possible for a star to produce energy,
but loses due to neutrinos still occur. As this happens electrons are
captured by the heavy nuclei, which in turn reduces the (electron
degeneracy) pressure, accelerating the in-fall. The temperature then
rises rapidly, allowing the photodisintegration of the iron nuclei.
\begin{equation}
  \label{eq:129}
  \gamma(100\,\mega\electronvolt) + ^{56}\ce{Fe} \to 13\, ^4\ce{He} + 4 \ce{n}
\end{equation}
This reaction absorbs around $2 \mega\electronvolt$ of energy per
nucleon. At this point the core collapses in near free-fall at around
$0.25 c$. As the process continues the temperature rises to the point
where helium disintegrates intro protons and neutrons, and free
protons capture free electrons to form neutrons. The collapse is
finally halted by the neutron degeneracy pressure when the density
reaches $10^{18}\,\kilogram/\meter^3$. This remnant is a neutron star.

The formation of the neutron star produces a shockwave as the outer
layers of the core continue to collapse, but this eventually stalls as
a result of photodisintegration and neutrino losses. After a few
milliseconds the gas stops expanding back outward, and the
proto-neutron star starts accreting matter at a rate of several solar
masses per minute. If this accretion is able to continue unabated no
supernova occurs, and within around a second the neutron star further
collapses into a black hole. If this does not occur within a few
seconds, however, neutrino losses force the remnant to become a
neutron star.

\subsection{Neutrino production}
\label{sec:neutrino-production}

During the collapse around 10 percent
($7\,\mega\electronvolt/\text{nucleon}$) of the total energy of the
explosion is lost to nuclear processes, while only 1\% of the
$3\e{46}\,\joule$ of gravitational energy is radiated as photons, and
a few percent as kinetic ejecta. The vast majority of the energy is
used to produce neutrinos. The burst of neutrinos from SN\,1987A was
observed by SuperKamiokande, with the burst lasting 13 seconds.

\section{Pair-instability supernovae}
\label{sec:pair-inst-supern}

Stars with a mass greater than 100 solar masses are still produced in
the galaxy, but these will normally have nuclear and
photometrically-powered pulsations which drive their mass loss. These
are, however, suppressed in extreme Population III stars, and maintain
most of their mass through to the end of their helium burning cycle,
resulting in a massive helium core. If the core has a mass greater
than 42 solar masses the star will experience pair-instability; the
result of high-energy gamma rays in the core after carbon burning has
completed, and this causes a pressure drop. For progenitor masses
between 100 and 130 solar masses this results in sequence of
disruptive pulsations which shed mass from the star until it is below
the instability limit, but for masses in excess of this the first
pulse will have enough energy to violently disrupt the star, producing
an explosion. These supernovae are credited with the production of a
nearly solar distribution of heavy elements up to zinc.

\section{Implosion to explosion}
\label{sec:implosion-explosion}

In order to turn the implosive nature of a supernova into the
explosive process which is observed energy is transferred from the
neutrino flux to the outer layers of the star, rather than the
shockwave produced by the formation of the neutron star. The expanding
material from the supernova will then form a shockfront as it travels
through the interstellar medium, however.

The explosion is the main source of the supernova's light curve. An
early peak is produced as the shock breaks through the surface of the
gas, producing strong UV radiation; the lightcurve will then plateau
as the hydrogen-rich envelope expands and cools and recombines. After
this the tail of the lightcurve is driven by the radioactive decay of
$\ce{^{56}Ni}$.

%%% Local Variables: 
%%% mode: latex
%%% TeX-master: "../project"
%%% End: 
