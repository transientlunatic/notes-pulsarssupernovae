The Galaxy is filled with a tenuous plasma, the interstellar medium,
which affects photons which travel through it. In normal observations
this effect isn't visible, since the majority of sources emit light
continuously, but for pulsars it is an important effect, and one which
must be corrected.

\section{Propagation of light in a plasma}
\label{sec:prop-light-plasma}

Consider a wave of angular frequency $\omega$, with the electric
field, with amplitude $E$, described by
\begin{equation}
  \label{eq:100}
  E = E_0 \exp(i \omega t)
\end{equation}
which propagates in a plasma of electron number density
$n~e$. Assuming that the protons are quasi-stationary, the electrons
rotate about them as
\[ x = x_0 \exp(i \omega t) \]
with an equation of motion
\begin{equation}
  \label{eq:101}
  E e = m~e \ddot{x} = -m~e \omega^2 x 
\end{equation}
Charge separation appears as a bulk polarisation, defining the
relative permittivity, $\epsilon~r$, of the plasma,
\begin{equation}
  \label{eq:102}
  P = n~e p = (\epsilon~r -1) \epsilon_0 E
\end{equation} 
for $p = xe$ the dipole moment of a proton-electron pair. Then, using
the equation of motion,
\begin{equation}
  \label{eq:103}
  \epsilon~r = 1 - \frac{n~e e^2}{\epsilon_0 m~e \omega^2}
\end{equation}
and the refractive index of the plasma is then
\begin{equation}
  \label{eq:104}
  \eta = \epsilon~r^{1/2} = \qty( 1 - \frac{f~p^2}{f^2} )
\end{equation}
where $f~p$ is the plasma frequency, 
\[f~p = \frac{1}{2 \pi} \qty( \frac{n~e e^2}{\epsilon_0 m~e})^{1/2}
\approx 9 \qty( \frac{n~e}{1\,\centi\meter^{-3}})^{1/2}. \] This is
the natural oscillatory frequency of the plasma. This leads to the
conclusion that for
\begin{itemize}
\item $f<f~p$ there are no propagating modes of EM radiation.
\item $f>f~p$ the group velocity is $c \eta$ and the phase velocity is
  $c/\eta$.
\end{itemize}
Local spatial variations in the density of the plasma lead to
scattering of waves as they propagate.

\section{Dispersion}
\label{sec:dispersion}

Since the refractive index of the ISM is frequency dependent it
follows that the resultant time delay is also frequency dependent,
\begin{equation}
  \label{eq:105}
  t = \int_0^D \frac{\dd{z}}{\eta(z) c}
\end{equation}
where $D$ is the distance along the line-of-sight to the source. So
long as $f~p \ll f$, this delay is
\begin{equation}
  \label{eq:106}
  \tau~D = \frac{e^2}{8 \pi^2 \epsilon_0 m~e c} \frac{1}{f^2} \int_0^D n~e(x) \dd{x},
\end{equation}
which is more normally expressed
\providecommand{\DM}{\operatorname{DM}}
\begin{equation}
  \label{eq:107}
  \tau~D = 4.15\e{3} \frac{1}{f^2~{\mega\hertz}} \DM\,\second
\end{equation}
where $\DM = \int_0^D n~e(x) \dd{x~{pc}}$ is the dispersion
measure. Dispersion of radio signals stretches the pulse out into
chirps across the band-width of the pulsar, with the highest
frequencies arriving earliest.

Pulsars allow the $\DM$ to be measured, and so provide a means to
measure the electron density throughout the Galaxy. This reveals the
typical density to be $n~e \approx 0.03\,\centi\meter^{-3}$.

\begin{table*}
  \centering
  \begin{tabular}{lllll}
    \toprule
    Component           & Temperature [$\kelvin$] & Volume fraction & Number density [$\centi\meter^{-3}$] & Species              \\ 
    \midrule
    Molecular clouds    & $20$--$50$              & $<1 \%$         & $10^3$--$10^6$                       & Molecular hydrogen   \\
    Cold neutral medium & $50$--$100$             & $1\%$--$5\%$    & $1$--$10^3$                          & Atomic hydrogen      \\
Warm neutral medium     & $10^3$--$10^4$          & $10\%$--$20\%$  & $10^{-1}$--$10$                      & Atomic hydrogen      \\
\midrule
Warm ionised medium     & $10^3$--$10^4$          & $20\%$--$50\%$  & $10^{-2}$                            & Electrons \& Protons \\
H\,II regions           & $10^4$                  & $10\%$          & $10^2$--$10^4$                       & Electrons \& protons \\
Hot ionised medium      & $10^6$--$10^7$          & $30\%$--$70\%$  & $10^{-4}$--$10^{-2}$                 & Electrons, protons, \& ions \\
    \bottomrule
  \end{tabular}
  \caption{The Interstellar medium}
  \label{tab:ism}
\end{table*}

\section{Interstellar Scintillation}
\label{sec:interst-scint}

In addition to dispersion it is possible to observe variations on the
scale of minutes in apparent pulse brightness; a process known as
interstellar scintillation. This is a result of the variations caused
by the movement of denser blobs of interstellar medium passing into
the line of sight.

A plasma with variations in the refractive index (often around $0.1\%$
in the ISM) will distort a planar wavefront. The excess refractive
index due to an electron density at $r$ is
\begin{equation}
  \label{eq:108}
  \Delta \eta(r) = \frac{e^2}{8 \pi^2 \epsilon_0 m~e} \frac{\Delta n~e(r)}{f^2} = \frac{r~e}{2 \pi} \lambda^2 \Delta n~e(r)
\end{equation}
for $r~e$ the classical radius of the electron.

This plasma can be approximated as being confined to a thin screen,
and composed of randomly placed, identical blobs of excess plasma
density. Each has a diameter of $a$, and the screen has thickness
$D$. Thus we expect a photon to encounter $D/a$ blobs, on average,
with an rms variation of $\sqrt{D/a}$. Each blob introduces a phase
change of $2\pi \Delta \eta a/\lambda$, so the perturbations across
the wavefront will be
\begin{equation}
  \label{eq:109}
  \Delta \Phi = r~e \lambda (Da)^{\half} \Delta n~e
\end{equation}
If $a \gg \lambda$ the blobs will refract the rays passing through
them, with a scattering angle
\begin{equation}
  \label{eq:110}
  \theta~s \approx \frac{\Phi}{2 \pi} \frac{\lambda}{a} = \frac{1}{2 \pi} r~e \lambda^2 \qty( \frac{D}{a} )^{\half} \Delta n~e
\end{equation}
and so a point source will appear to be broadened. 

Different rays from the blurred source will take differing amounts of
time to reach the observer, and so a pulse will be broadened. If the
distance from the screen to the observer is $z$ this broadening,
$\tau~s$ is
\begin{equation}
  \label{eq:111}
  \tau~s = \frac{z}{c} (1-\cos(\theta~s) ) \approx \frac{z \theta~s^2}{2 c}
\end{equation}
if $\theta~s \proptp \lambda^2$ it follows that $\tau~s \propto
\lambda^4$. Thus a thin pulse is broadened to an exponential decay
with an intensity profile
\[ I(t) \propto \exp(- \frac{t}{\tau~s} ) \] multiple scattering
events smoothes this.
 
The full evolution of the wave is described by the Fresnel diffraction
formula,
\begin{equation}
  \label{eq:113}
  \Psi(R) = \frac{e^{-i \pi/2}}{ 2 \pi r^2~F} \iint \exp( i \phi(r) + i \frac{\abs{r-R}^2}{2 r~F^2} ) \dd[2]{r} 
\end{equation}
for $r~F$ the Fresnel scale,
\[ r~F = \qty( \frac{\lambda z}{2 \pi} )^{\half} \] This integral is,
in principle, over the whole screen, but in practice the effect in the
first Fresnel scale unit dominates. If the disturbance is small $\ll
\pi$ the scattering is described as \emph{weak scattering}; if there
are large variations it is \emph{strong scattering}.

In the strong scattering regime the scattering generates a large
variation over the Fresnel scale, demanding a new phase-stationary
scale, $r_0$. Each blob diffracts radiation by the scattering angle,
$\theta~s \approx 2 \pi \lambda / r~{diff}$, with $r~{diff} = r_0$
being the diffracive scale. The observer sees the radiation from the
blobs on a scale of $r~{ref}= z\theta~s$, the refractive scale.

These are related to the Fresnel radius as $r~{diff} r~{ref} =
r^2~F$. In the weak scattering regime the effect is limited to a
single scintillation mode rather than diffraction and refraction.

If the radio source is sufficiently small and band-limited the phase
screen will be illuminated by spatially coherent radiation, and the
overlapping scattered waves will produce a strong and random
interference pattern with a scale size of $r~{diff}$. In order to
maintain this interference pattern it is necessary to restrict the
bandwidth to the inverse of the broadening time,
\[ \Delta f \approx \frac{1}{2 \pi \tau~s} \] which is the
decorrelation bandwidth of the scintillations.

\section{De-dispersion}
\label{sec:de-dispersion}

In order to resolve a pulse profile it is necessary to remove the
effects of dispersion from the signal across the passband (the set of
wavelengths which are recorded by the radio telescope). In order to do
this the passband is split into smaller sets of wavelengths, and
passed through a bank of filters to realign the passband, producing a
coherent pulse. This is more normally done inn software now, after
first digitising the passband. However, the digitisation of the
passband allows a more sensitive method to be used, coherent
de-dispersion.




%%% Local Variables: 
%%% mode: latex
%%% TeX-master: "../project"
%%% End: 
