The majority of galactic pulsars are in the galactic plane, and have a
high velocity distribution, with the average velocity of pulsars
around $700\,\kilo\meter\,\second^{-1}$, and $10\%$ having velocities
over $1000\,\kilo\meter\,\second^{-1}$. As a result a large number
have galactic escape velocities.

\section{Millisecond pulsars}
\label{sec:millisecond-pulsars}

Millisecond pulsars form a distinct population on the P-Pdot diagram,
and are old pulsars which have undegone a process which has spun them
up. These pulsars generally have rotation periods under
$10\,\milli\second$, but up to $30\,\milli\second$. The majority of
neutron stars associated with x-ray sources are members of binary
systems, but only around 10\% of millisecond pulsars are, a fact
normally explained by the kick provided to the neutron star by the
supernova explosion. 

The transfer of mass between two components of a binary system is the
mechanism which causes a pulsar to spin-up and form a millisecond
pulsar. Angular momentum is transferred from the motion of the binary
orbit, and are recycled. They have characteristic ages in excess of
$10^9$ years, thanks to their very slow spin-down rate. The first
millisecond pulsar to be discovered was PSR\,B1937+21, which was a
discrete radio source which had no obvious periodicity until
technology allowed it to be sampled at a period below
$1.6\,\milli\second$. Many millisecond pulsars are strong gamma
emitters, some producing no obvious radio emission. The spin-up rate
is limited by the rotation rate of a star.

\section{Binary pulsars}
\label{sec:binary-pulsars}

\section{X-ray pulsars}
\label{sec:x-ray-pulsars}

Radio pulsars are powered by the rotational kinetic energy of the
pulsar, but in systems with a binary companion an x-ray pulsar can
form, using energy from material accreted from the
companion. In-falling gas is channelled onto the polar regions of the
neutron star, and creates hotspots at the points where it meets the
surface.

Cen X-3 is a good example of a Roche-lobe overflow system, as a LMXB,
this object shows a measurable spin-up. Vela X-1 is part of a HMXB,
and is powered by accretion of material from the stellar wind of a
massive (O/B-type) companion. Neutron stars in elliptical orbits about
Be (gaseous disk) stars will only produce pulses as the neutron star
passes through the disk, so will produce pulses after a given
interval.

Some x-ray pulsars, AXPs, are anomalous, and are not powered by the
accretion of gas, or the rotation of the neutron star. These are slow,
with periods between $5$ and $12$ seconds, and show strong x-ray
outbursts. their spin-down rates imply a magnetic field strength of
$10^{13}\,\text{G}$--$10^{15}\,\text{G}$, and this seems to be the
source of the high x-ray luminosity.

Likewise, soft gamma ray repeaters, SGRs, produce repeating bursts in
the gamma regime, with flarings lasting a few minutes, with a periodic
structure in the decay which imply long (seconds) rotation
periods. These objects are believed to be magnetars, and flares are
the result of sudden reconfiguration of the magnetisation as a result
of the fracturing of the surface after a starquake. The slowdown rates
of SGRs are very high, which is also consistent with magnetar-level
fields.

\section{Magnetars}
\label{sec:magnetars}

These are newly-born and rapidly rotating neutron stars, with an
intense internal dynamo effect which increase the magnetic field
strength by factors of $1000$, up to $10^{11}\,\tesla$. This may exist
in as many as $10\%$ of neutron stars, with the superconducting core
retaining the field.

In a magnetar the rotation falls from $\sim 1000\,\hertz$ to $\sim
300\,\hertz$ in a period of around $10\,\second$, so $90\%$ of the
energy is dissipated, potentially being the source of much of the
heavy nucleogenesis in the universe. A combination of the rapid
braking and the resultant short lifespan ($\sim 10^4$ years) leads to
only 14 having been observed, although as many as $10^7$ may exist in
the galaxy.

\section{RRATs}
\label{sec:rrats}

RRATs are rotating radio transients, which produce transient radio
flares with around $2$ to $30\,\milli\second$ durations, separated by
periods of four minutes to three hours, and are likely to be neutron
stars with insufficient magnetisation to become magnetars, but the
acceleration mechanism which produces radio emission is not completely
suppressed.

\section{Gamma ray bursts}
\label{sec:gamma-ray-bursts}

Gamma ray bursts are transient sources of gamma radiation which do not
repeat, producing harder gamma rays than SGRs. There are two
categories: short duration (<2 second) bursts are thought to be the
result of coalescence of binary neutron star systems, producing beamed
(10 degree beamwidth) radiation, and can be extragalactic; long
duration bursts are believed to be the beamed emission from a massive
star collapsing into a black hole.

%%% Local Variables: 
%%% mode: latex
%%% TeX-master: "../project"
%%% End: 
