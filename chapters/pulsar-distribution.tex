By making individual timings of pulsars it is possible to measure
their period's variation over time,
\begin{equation}
  \label{eq:114}
  P(t) = P(t_0) + \eval{\dv{P}{t}}_{t_0} (t-t_0) + \eval{ \dv[2]{P}{t}}_{t_0} \half (t-t_0)^2 + \cdots
\end{equation}
The first two coefficients, P and Pdot, are found by fitting a second
order polynomial to the pulsar timings; the higher coefficients are
small and dominated by timing noise. Figure \ref{fig:ppdot} shows the
P-Pdot diagram; the pulsar world's equivalent of the Herzprung-Russel
diagram.

%%% Local Variables: 
%%% mode: latex
%%% TeX-master: "../project"
%%% End: 
