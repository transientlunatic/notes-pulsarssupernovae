A mean pulse profile is composed of the superposition of a large
number of sub-pulses with an approximately gaussian profile, with
random position and strength. The integrated profile is then
reflective of the statistics of the sub-pulses. The sub-pulses
themselves can show complex and organised structure, while some
sub-pulses are not observed; periods known as nulling.

\section{Micro-pulses}
\label{sec:micro-pulses}

Each pulse from the pulsar is composed of smaller micro-pulses; the
Crab pulsar shows a remarkably short micro-pulse structure within the
giant pulses. The sub-pulse proile which is created by these
micro-pulses shows a strong polarisation structure.

The sub-pulse structure can be explained in terms of the structure and
evolution of the pulsar beam; different models suggest that this could
either be the result of nested cones of radiation intensity, or
patches of radiating material across the surface of the pulsar.

Different slices through the beam profile will lead to differing
sub-pulse profiles, and so they evolve over time, and may eventually
precess out of sight.

\section{Magnetosphere}
\label{sec:magnetosphere}

Even despite the intense strength of the gravitational field close to
a pulsar the Lorentz force on a charge still vastly exceeds it.

\begin{align*}
  F~{em} &= e(\vec{v} \times \vec{B}) \\ &= e R \Omega B \\
F~g &= \frac{GMm~e}{R^2} \\
\frac{F~{em}}{F~g} &= \frac{e \Omega B R^3}{G M m~e}
\end{align*}
Now, putting $B=10^9\,\tesla$, $\Omega = 2\pi / 1\,\second$, and $M = 1.4M_{\odot}$, then
\begin{equation}
  \label{eq:112}
  \frac{F~{em}}{F~g} \approx 10^{12}
\end{equation}
As a result charges move as if there was no gravity whatsoever, and
follow magnetic field lines; residual charges on the surface of the
neutron star are stripped, and a charged magnetosphere develops, which
corotates with the neutron star as if it were a solid body. Charges
can also move freely through the neutron star, acting like a
superconductor.

The corotation is limited by special relativity to the light cylinder
of radius 
\begin{equation}
  \label{eq:123}
  R~L = \frac{c}{\Omega}
\end{equation}
Field lines which are within the light cylinder are closed, but
outside it they are open. Modelling this structure is difficult, but
charges should flow until the Lorentz force is balanced by an electric
force, when
\begin{equation}
  \label{eq:124}
  \vec{E} + (\vec{\Omega} \times \vec{r}) \times \vec{B} = 0 
\end{equation}

\subsection{The polar cap model}
\label{sec:polar-cap-model}

Regions of the magnetosphere where the force-free state cannot be
maintained are gaps. One exists above the magnetic polar cap, where a
depleted charge concentration will produce a strong net force on any
charge. Photons are then accelerated, and interact with the $B$ field,
producing electron-positron pairs, and a cascade of radiation close to
the surface of the neutron star. This is believed to produce the
coherent radiation of the radio beam.

Close to the light cylinder there is a second gap; the outer gap,
where there is a difference of around $10^{14}\,\volt$, close to the
null line, $\vec{\Omega} \vdot \vec{B} = 0$ separating regions of
opposite charge. Here the field is weaker than at the poles, and pair
production is harder. Radiation from this gap appears as synchrotron
emission, and curvature radiation, in the optical, x-ray, and gamma
regimes.

\subsection{Relativistic beaming}
\label{sec:relativistic-beaming}

Material which is close to the light cylinder moves at close to the
speed of light, and the effects of relativistic beaming concentrate
the radiation into a beam of angular width around $1/\gamma$, As the source chases the radiation the beam width is shortened further to $1/\gamma^2$, and so the beam sweeps over the observer in a time 
\begin{equation}
  \label{eq:125}
  \tau \approx \frac{1}{\Omega \gamma^3}
\end{equation}

The beam has a brightness temperature greater than any achievable from
random processes,
\[ T~b = \frac{Bc^2}{2 \nu^2 k} \sim 10^{30}\,\kelvin \] Implying
coherent emission, where power increases with the square of the number
of emitting particles.  The process probably relies on electrons
bunching as they are accelerated. Radio pulses widen at lower
frequencies, possible because we can see emission from higher in the
emission cone at lower frequencies, but the reason for this is an open
problem in plasma physics.

%%% Local Variables: 
%%% mode: latex
%%% TeX-master: "../project"
%%% End: 
